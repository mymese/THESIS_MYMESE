%
% This chapter is included in the template directory to demonstrate
% the figure, table, musex, and equation environments.  We'll also
% demonstrate the \cite, \label, \ref, and \pageref macros.
%
% If you wish to test the features demonstrated in this file, then
% you should uncomment the '\usepackage[dvips]{graphics}' line near
% the beginning of 'mythesis.tex', and you should also uncomment the
% '\input chapter2' line.
%
% In order to resolve all the references (including \cite), you will
% also need to uncomment the '\bibliographystyle{plain}' line and the
% '\bibliography{myrefs}' line (found near the end of mythesis.tex).
% Then process the file as follows:
%
%   1. latex mythesis
%   2. bibtex mythesis
%   3. latex mythesis
%   4. latex mythesis
%

\chapter{This is a Chapter for Testing}

\section{Referencing Equations}

Leonhard Euler was an astounding mathematician whose pioneering work
in power series helped to develop the field of mathematical analysis.
Equation~\ref{eq:euler-id} on page~\pageref{eq:euler-id} is known as
\textit{Euler's Identity}, what physicist Richard Feynman called ``the
most remarkable formula in mathematics''.
\begin{equation}
  e^{i\pi} + 1 = 0
  \label{eq:euler-id}
\end{equation}

\section{Building a Simple Table}

Refer to table~\ref{tab:peeps-like-tables} to review the percentage
of the population who like tables and graphs.\footnote{This
information is completely invented.}
\begin{table}
\caption{People who like tables and graphs, broken out by
  gender.\label{tab:peeps-like-tables}}
\centering
\begin{tabular}{lrrr}
\textbf{Gender}&\textbf{Likes}&\textbf{Dislikes}&\textbf{Undecided}\\
Male&51\%&25\%&24\%\\
Female&48\%&27\%&75\%\\
Undecided&12\%&25\%&63.442\%
\end{tabular}
\end{table}

\section{I Hate Music, But I Love to Sing}

I think every college soprano learns this song at one point or another
(that is, the song that titles this section).
Example~\ref{mus:beethoven} on page~\pageref{mus:beethoven} is one
that every bass should learn at some point in his career.  It is a
\textit{tour de force} of choral and orchestral literature.

The \verb+\includegraphics+ macro will insert the file containing the
figure here.  The figure is in a directory called \texttt{music}, and
the file is called \texttt{freude.eps}.  (\verb+\includegraphics+
tests for several commonly used file extensions.)  The kinds of
figures which may be included will depend on the driver that was
specified with the \verb+\usepackage[driver]{graphics}+ command.  You
will need to read some documentation on the \texttt{graphics} package
to learn about what options are open to you.

\begin{musex}
\begin{center}
\includegraphics{music/freude}
\end{center}
\caption{Ludwig van Beethoven, \textit{Symphony No.~9 in D minor},
  Op.~125\label{mus:beethoven}}
\end{musex}

And no choral person worth his or her salt would turn down the
opportunity to sing this rousing work at least once.
See section~\ref{sec:citations} to learn more about citations.

\section{Here a Cite, There a Cite\label{sec:citations}}

I'm testing the waters here:  can Whissel~\cite{whis86} cite himself
without reproach?  Or will he be proached at all?  Actually, one should
refer to the Lamport book~\cite{LatexBook} for more information on
citations and bibliographies.  The Oetiker, et al., manual~\cite{NotShort}
makes but brief mention of the topic.

\section{Including Figures}

If I have external figures, I may include them if the \texttt{graphics}
package has a driver that can handle the figure type.  If my
graphics-generating software can produce PostScript images, then we're
probably OK.  If you'll notice, Figure~\ref{fig:piechart} is a pie chart.

For this figure, I've told \verb+\includegraphics+ to look in the
folder \texttt{figures} for a file called
\mbox{\texttt{pie.}\textit{something}}.  Because
\verb+\usepackage[dvips]{graphics}+ specified the \texttt{dvips}
driver, then PostScript figures are usable.  If you intend to use
\texttt{pdflatex} to generate PDF files directly, you will need to
select a different driver, and then your figures will need to be other
PDF files, not PostScript figures.

\begin{figure}
\begin{center}
\includegraphics{figures/pie}
\end{center}
\caption{This figure is about nothing terribly important, but at
  least it's in pie-chart form.  This immediately makes it more
  interesting, and possibly tasty.\label{fig:piechart}}
\end{figure}

