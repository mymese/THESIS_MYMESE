\chapter{INTRODUCTION}

%Gordon: 

% Introduction describes the setting for what you want to do. Tell a story. Why are you doing this work? What are the issues? What are the objectives?

% describe what blender is, how popular it is, what it does, why people like it, its advantages, etc.

% Why flocking? What can one do with flocking? Surely you read articles? Armies, animals, population, better games, etc., etc. Why are we doing flocking? What can we achieve? Why Blender?

%What have you done? What are you hoping to do with your implementation?

%%%%%%%%%%%%%%%%%%%%%%%%%%%%%%%%%%%%%%%%%%%%%%%%

% why people like games?
% creating games
% what a game must have in order to be a good game?

Computer games have been around since the 1950s\cite{computerVideoGamesHistory}. They were available in the market place by the 1970s and they were very popular. \EDIT{Since then, there has been a veritable explosion of games as new hardware devices came on the market, as computers following Moore's law continuously increased their computational power, and as gamers made increasing demands on game requirements. }{More and more games have been developed since then.} 
We have now entered the seventh generation of video game consoles\cite{seventhGenerationGames}. Video consoles such as the PlayStation, Xbox, and Wii are the most popular ones. \cyan{Just last November (6 months ago)}{}, Xbox released their new control-free body play experience called Kinect\footnote{http://www.xbox.com/en-US/kinect}. This new hardware \EDIT{is again changing}{changed} the game industry by allowing the player to control the game experience with their own \EDIT{body movements rather through the intermdiary of some external device. }{moves. } \EDIT{This type of new hardware adds realism to the game in the sense that there are no artificial input devices. }{The user interacts with the game console more nt Hardware like this make the developing of games more fun.}

\textit{How to know if a game is a good game?} A game, to be attractive to players, should have at least three important \EDIT{ingredients}{aspects}\cite{bookGameKit2}. First is the toy aspect, which refers to immediate fun that you can have playing the game. You would not need to read any manual to play it, you just grab the controller or the mouse and start \EDIT{exploring}{playing it}. These types of games are more intuitive. Second is the immersive aspect of the game: \EDIT{the player forgets he is in a game and "lives" the game.}{ This is when you are playing a game and you completely forget that you are playing.} Realism is a very important factor. \EDIT{}{in here.} An example of immersive games are simulator games. Third is the goal aspect. This games give you a goal that you have to achieve, and involve some kind of strategy and planning. When creating games it is important to create balance between these aspects. But, \textit{how to create the games?} Games are created using game engines.
\mmy{NO REFERENCES IN THE ABOVE PARAGRAPH?}

% different software to create games available
% why Blender?

\EDIT{Currently,} There are many commercial and free game engines available \EDIT{for users interested in their own game development}. Our interest \EDIT{in games is to help improve education at all levels of schooling, from K-12 up to college.}{} \EDIT{We feel that whatever the end results, the software should be widely available, low or no cost, and should run on the major platforms (Linux, MacOSX and Linux}{}.  
\EDIT{}{such, i  in one for educational purposes that have the capability of been extended (open source), that is cross-platform between Windows, MacOSX and Linux, and that also can support simulation with real physics.}

\mmy{NO references? How about Wikipedia's list of game engines. there are other lists available. Use Google.}

% benefits and difficulties of using Blender
% Blender contents

\EDIT{During the development of a new Game Design course created as part of the undergraduate curriculum of the Scientific Computing program, one of the 
requirements was that software used by the students satisfy the requirements
above, be free of cost, have a strong and active user community, an easy to 
learn scripting language, a game engine and powerful modeling tools.}{Dr. Erlebacher searched for software to teach his \textit{Introduction to Game Design} class for the first time, he found that Blender\footnote{http://www.blender.org} met all the requirement mentioned above, plus it was free.} 
\EDIT{This led to the choice of Blender as a software platform.}{}  
Blender was created 1993, initially to create 2D and 3D content, \EDIT{but it grew to include modeling, texturing, animation, particle simulation, rendering, and game creation, among its many capabilities}{}. The Blender Game Engine has powerful tools to create games without the need \EDIT{for}{of any} explicit programming, \EDIT{although Python is also available for power users}{}. Blender has strong support from the gaming industry and there is a lot of documentation available \EDIT{(http://www.blender.org)}{}.  

% Blender Particle Systems
% issues between the Particle systems and the Game Engine

There are some additional tools \EDIT{available to the blender modeling and animation  area but not to the game engine}{that are not available for the Blender Game Engine, including particle physics and cloth simulation}.  \EDIT{}{those tools are only available for simulations or animations, for example.} The particle systems can simulate particles with different types of physics i.e. Newtonian, or boids \footnote{Entities that have flocking behavior.} and keyed physics. 

% CHECKED
% why flocking?
% summary of flocking

% define the problem and mention the approaches to solve it

When particles act according to Newtonian physics, they are subject to Newton's laws: force equals mass times acceleration. \EDIT{Particles are given an initial linear and angular velocity. The motion of each particle is determined from the force acting on it. For example, a collection of particles could be subject to pressure and viscosity forces.i}{} \EDIT{Another system of particles in Blender model flocking behavior.}{available is the Boids particle system.} \EDIT{In this case, the individual particles are called boids, and they are subject to simple rules that determined their velocity, rather than a force.}{In general, the rules  hi, systems follow some basic rules and behaviors} These rules are used to simulate flocks, schools, herds, and swarms of different entities.

%CHECKED
This thesis project focuses on the Boids system. Boids can be defined as entities, in most cases animals, that have an emerging behavior caused by following a set of rules called steering behaviors. This behavior is called \textit{flocking}. The word Boid was first introduced by the pioneer of flocking, Craig Reynolds in his 1987 \EDIT{}{published} paper Flocks, Herds, and Schools: A Distributed Behavioral Model\cite{craig1}. The model presented by Reynolds has been used by many scientists interested in this research area. \EDIT{It will be discussed further in the next chapter.}{}
\EDIT{}{\footnote{More information about this is going to be discussed in the next chapter.}}.

We will implement Reynolds' basic flocking model into 
\EDIT{}{described by Reynolds in his 1987 paper, the flocking algorithm to be used in} 
the Blender Game Engine. \EDIT{Although there have been many publications on flocking since the original seminar paper, the algorithm has not changed much.}{ , Sure you are thinking that there has been many publications after the first approach, \textit{Yes}!  Since the initial approach, the algorithm have not changed much,} \EDIT{Most of the modifications relate to the computation of the steering behaviors.}{that distinguish each publication are basically in the how they compute the steering behaviors.}

\EDIT{In order to introduce
the flocking algorithm into the Blender Game Engine, Blender's source code had to be  modified}{}, \EDIT{and Blender had to be rebuilt.}{implies that eventually Blender has to built again.} \EDIT{Much effort was spent ensuring that Blender remained portable across platforms.} All the modification in this thesis were based on  The property of the portability of Blender should be kept when taking decisions on the Blender release 2.57. 
%how to modify Blender's source code. This was not and easy job. 
%This thesis was based 

%CHECKED

% may need a connection sentence between this paragraphs

\EDIT{Up until this point, the developers in the Blender community restricted
the software to a pure CPU implementation. The physics engine and other components of Blender were not ported to the GPU since portability was a major priority  and the dominant GPU language, CUDA, did not run on the AMD graphics cards.}{} 
%The current version of Blender is CPU-based	
\EDIT{
Since the advent of OpenCL, it became possible to write GPU  software portable across the dominant computer architectures. As a result, we have implemented a particle system for flocking that runs on the GPU, taking advantage of the CPU-based facilities already offered by Blender, such as object manipulation, modeling, and its current game engine. 
As the game levels  become increasingly complex or the physical effects too 
computationally demanding, Blender starts dropping frames to try and 
maintain the desired animation speed. Since we expect to handle thousands
or tens of thousands of boids, it was imperative to leverage the power of
the graphics processor, which offers vector-based operations and light-weight multithreading. In this thesis, we implemented the boid system in OpenCL.}{}
\mmy{Need a reference to OpenCL and a reference to GPU architecture}

\EDIT{}{
%Even though the Blender Game Engine tries to display games at a speed of 60 frames per second\cite{bookGameKit2} if the CPU or graphics card is not good enough the engine will start dropping down some frames in order to get into a state in which the speed of all objects is maintained. Therefore, the computationally-intensive parts of the code were accelerated and the scenes were efficiently render. These was achieved by doing those computations in the Graphics Processor Units (GPUs) which are computation-dedicated devises that can perform vector operations in the data. GPUs have been very popular for the last years in different research areas.  }

\EDIT{}{
The programming language chosen for the GPU does not depend on an specific type of card. Blender already have the platform portability property and portability between different graphics cards must be maintain too. That is why OpenCL was selected as the language to be used for coding the computationally-intensive parts of the code that was accelerated. CUDA is a more popular GPU programming language but it works only on NVIDIA graphics cards. OpenCL ensures the portability between different GPUs.
}

%Why are you doing this work? What are the issues? What are the objectives?
%What have you done? What are you hoping to do with your implementation?

\EDIT{ Given that Blender is still evolving, and will for the foreseeable future, it was imperative to make our changes to blender in a way to minimize the
code footprint on the Blender source code.}{} Thus, our
GPU code is maintained outside the Blender source tree. My colleague Ian Johnson\footnote{http://enja.org/} developed a library named \textit{RTPS} (Run-Time Particle Systems) that defines particle systems (based on the Smoothed-Particle Hydrodynamics) that can be used inside the Blender Game Engine. SPH can simulate in real time real-time fluids with a few hundred thousand particles. 
We extended \textit{RTPS} in order to implement flocking behavior. The \EDIT{steering}{three basic} rules plus some other additional behaviors were implemented inside \textit{RTPS}, which also implements all the rendering routines.
\mmy(YOU NEVER SAID THERE WERE THREE RULES yes}

% CHECKED
An interface to the Blender Game Engine was developed to link with the RTPS library. This interface is defined via Blender's modifier framework. 
%This modifier was called RTPS. 
It allows us to applied properties to objects inside the Blender Game Engine and define interactions between the various Particle Systems and Blender's own objects.
%Both the RTPS library and the Blender modifier were expanded in order to enhance the capabilities of the main implementation done by Ian Johnson.

% guide to the reader of the content of the thesis

\EDIT{The remainder of the thesis is structured as follows: }{This thesis continues as follows}  In Chapter, we  describe some related work in the area of flocking. Chapter 3 describes the flocking algorithm used and the specific rules implemented. Chapter 4 provides details on the structure of the RTPS library. In Chapter 5, the Custom Modifier that was developed for Blender and some other properties defined to be able to interact with other Blender objects are presented. Chapter 6 discusses the results with benchmarks and presents some demonstrations. Conclusions are found in Chapter 7.  And finally, Chapter 8 discusses possible Future work. 

