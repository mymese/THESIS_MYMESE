\chapter{RELATED WORK}\label{chap2}


%%%%%%%%%%%%%
% Introduction
%%%%%%%%%%%%%
%\section{Introduction}
Flocking is a nature-inspired behavior mostly seen in animals (birds, fish, large herds, ants, etc.) The pioneer of flocking simulations, Craig Reynolds, defined flocking as \textit{the result of the interaction between the behaviors of individual birds}\cite{craig1}. Since Reynolds' publication in 1987, the field of flocking started to grow rapidly. %Research work related to flocking started to been developed by mid 1980s. 

This chapter discusses some related work on flocking. We first discuss the early work of Craig Reynolds and some of the research that followed. We then present a brief summary of GPU computing, and how flocking algorithms have been dramatically accelerated. The final section talks about educational games and how they might be served by the incorporation of a flocking module. 

%%%%%%%%%%%%%
% Computational Intelligence
%%%%%%%%%%%%%
%\section{Computational Intelligence}

% Swarm intelligence
%\subsection{Swarm Intelligence} 
%Swarms are seen in nature mostly as large groups of small insects \cite{?}. Each entity of a swarm performs a simple role that evolves into a complex behavior as a whole. The emerge of this complex behavior goes beyond swarm since it can also be seen in social animals like birds and fishes. 

% take a picture of a flock to add it here 

%Each entity of a swarm will follow a set of simple rules which depends from the entities of the swarm. Every time the rules are changed the environment conditions are evaluated. Since this process is sometimes very simple researches started trying to model it. Therefore, a new discipline in the area of artificial intelligence started to growth.

%In 1989 Gerardo Beni and Jing Wang introduced the term Swarm Intelligence (SI) to the Artificial Intelligence (AI) community. SI was born in order to understand the biological insights about the ability of social life insects to solve their everyday problems \cite{BioPrinciplesSI}. As many areas of science, researchers give different definitions to it. XXXX and XXX defined SI as an AI discipline that studies the self-organized behavior of multi-agent systems \cite{?}. 

%Lets explain what is self-organized behavior and multi-agent systems are and were are they used for. Self-organized behavior.

% mention ACO and PSO

% Evolutionary computation
%\subsection{Evolutionary Computation}

% mention GA


%%%%%%%%%%%%%
% Flocking
%%%%%%%%%%%%%
\section{Flocking}
Flocking became very popular in the late 1980s, when, inspired by the behavior of birds, Craig Reynolds developed a behavioral model that simulates self-organizing boids\footnote{A boid is synonymous with one entity taken from a flock.}. This led to much research and a wide range of applications. Some of these applications will be discussed in the following section.  

% Early work by Craig Reynolds
\subsection{Original Boids Model by Craig Reynolds}
Flocking became very popular once Craig Reynolds published his original \textit{Boid} model\cite{craig1}. Reynolds introduced the word \textbf{boid} to denote a generic \textit{simulated bird-like, \textbf{bird-old} object even when they really represent other entities such as a fish in a school.} The original model was inspired by the flocking behavior of birds. The contributions in the original paper inspired many researchers to become interested in studying this type of behavior occuring in the natural environment. 

His flocking model was inspired by observations of birds in flight. Several basic assumptions led a simple set of rules that could approximately model the coarse characteristics of bird flight, combining randomness with obvious correlation between the flight trajectories of individual entities.  

The flight of natural flocks is defined by a balance between the desire to stay close to each other and the need to avoid collisions. This balance leads to the definition of the Boid model. The first rule defined by Reynolds is \textit{collision avoidance}: each boid adjusts its trajectory to move away from neighbors that are too close;  the second rule is \textit{velocity matching}: in the absence of other considerations, the boids want to move at the same velocity as their neighbors. In other words, the flock wants to maintain its structure. The last rule is \textit{flock centering}: boids tend to form clusters by moving closer to their flockmates. 

Collision avoidance is based on relative positions while velocity matching is based only on the velocities; they are therefore complementary rules. Flock centering steers each boid towards the center of a cluster of local neighbors. 

From an implementation point of view, each boid stores its internal state and evaluates each of the rules individually based on the flockmates in some prescribed neighborhood. In practice, the neighborhood is defined as a spherical zone around the boid's local origin. When Reynolds ran his experiments, initial positions, velocities, and various parameters were initialized with random values. 

When avoiding collisions with obstacles, the boids react in a way that depends on the force field that surrounds them. Naturally, the boids only consider objects that are in front of them. The naive implementation of the original Boids model had an operation count of $O(N^2)$, where $N$ is the number of boids in the flock. This reflects the dominant cost of the neighbor search. Reynolds also performed a $O(N)$ parallel implementation.

% Current work
\subsection{Previous Flocking Work}\label{currentwork}
Since the publication of the original Boids model, many papers on flocking have been published. Research has focused on expanding the list of the steering behaviors\cite{craigSteeringBehaviors}, enhancing the neighbor search\cite{spatialSwarms}, analyzing the different types of  boid formations\cite{lineFormations}, or just applying flocking or modified flocking algorithms to a specific research problem.

The second big contribution of Reynolds came out in 1999\cite{craigSteeringBehaviors}, in which he introduced additional steering behaviors to define autonomous characters. Autonomous characters are agents in the animations or games that do not need to be controlled explicitly because they improvise their actions and moves. In games, these agents are called non-player characters. The motion behaviors of an autonomous character can be divided into three layers: action selection (strategy, goals, and planning), steering behaviors (path determination), and locomotion (animation). Reynolds focussed on the study of the second layer. The steering behaviors presented were:
\begin{enumerate}
\item \textbf{seek}: boids steer towards a static target in global space
\item \textbf{flee}: the inverse of seek, boids steer away from the target in global space
\item \textbf{pursuit}: similar to seek but the target is a moving object
\item \textbf{evasion}: similar to flee but the target is a moving object
\item \textbf{offset pursuit}: steer the path to pass \textit{near to} but not \textit{directly to} the moving object
\item \textbf{arrival}: similar to seek, but the character is far from the target
\item \textbf{obstacle avoidance}: gives the character the ability to maneuver in the environment while not colliding with the obstacles 
\item \textbf{wander}: random steering
\item \textbf{path following}: steer along a predetermined path
\item \textbf{wall following}: variation of path following, approach a wall and maintain a certain offset from it
\item \textbf{containment}: variation of path following, motion is restricted to a region
\item \textbf{flow field following}: steers the position of the character in direction to the flow
\item \textbf{unaligned collision avoidance}: prevents collision between characters that are moving in arbitrary directions
\item \textbf{separation}: maintain certain separation from others nearby
\item \textbf{cohesion}: steers towards the center of nearby characters
\item \textbf{alignment}: steers towards the average heading of nearby characters
\item \textbf{flocking}: combination of separation, alignment, and cohesion
\item \textbf{leader following}: one or more characters follow another moving character (leader)
\item \textbf{interpose}: try to insert a character between two moving characters
\item \textbf{shadow}: approach a character and then use alignment to match speed and heading
\item \textbf{hide}: identify the position of a target that is on the opposite side of an obstacle and  steer towards it using seek
\end{enumerate} 
 
The above behaviors can be combined to produce more complex behaviors. 
In 2000, Reynolds focussed on the interaction of large groups of autonomous characters 
in real-time\cite{craigInteractionGroups}.

Von Mammen\cite{spatialSwarms} performed a study on the time evolution of flock topology, which depends on the dynamic change of the neighbor search. Marina Klotsman and Ayellet Tal\cite{lineFormations}  classified the different flock formations into two groups: \textit{cluster} and \textit{line} formations.

% TODO: add more references in general flocking
%\mmy{THAT IS IT? NOTHING MORE TO SAY?. YOu only mention two papers aside from Reynolds? Nothing more to be said?} 

\subsubsection{Applications}
There are numerous applications of flocking ranging from bio-inspired systems to clustering and robotics. We present some pertinent results from these application areas. One of the nature-inspired applications of flocking is prey/predator\cite{gems2}. Prey and predator simulations have been done, for example, to study the transitions of a prey flock while they escape from the predator\cite{preyFlock}. Also, crowds simulations, by definition a flocking phenomenon, have been considered\cite{crowdsPS3}. 

In the area of robotics, Yang et al.\cite{flockingRobots} developed a collision avoidance flocking algorithm based on the use of convex objects as obstacles. They defined four states for a robot: wait, observe, compute, and move. The life of these robots was a defined sequence of cycles of these states. The practical objective of the research was enhanced rescue operations after severe earthquakes and space exploration, two use cases in which humans would have a difficult time. The authors found that the algorithm they developed can efficiently adapt to complex environments. Lindh\'{e}\cite{flockingRobotsThesis} also focused in using a flocking algorithm to displace a group of robots. The implementation was actually tested on real robots. Lindh\'{e} prioritized the different components of his algorithm. The highest priority was safety which implies collision avoidance, next was goal convergence, followed by cohesion. He found that when robots were moving in open fields they stay clustered in a formation that facilitates communication between the robots. 

Unmanned air vehicles (UAV) can be related to robotics. Crowther\cite{flockingUAV} focuses in civil and military applications of UAVs. Crowther determined that cohesion and alignment were sufficient to generate the desired behavior. Another investigation of UAVs based on flocking behavior was conducted by Ryan et al.\cite{UAVControl}. They implemented collision avoidance with respect to other UAVs and obstacles. 

In 2006 Cui et al.\cite{document1} presented a flocking-based approach for document clustering. In this study each document is represented by a boid. A fourth rule was added to drive the flocking behavior based on the features of similarity and dissimilarity between the documents. This additional rule is what made the actual classification possible. 

Finally, we mention a variation on the flocking concept.  Influence maps were used to implement the flocking behavior and compared to the traditional implementation of flocking\cite{flockingInfluenceMaps}. The influence maps procedure outperforms the traditional procedure for large flocks in general. Many other applications of flocking are currently investigated, although we only touched on a few. 

%%%%%%%%%%%%%
% GPU Computing
%%%%%%%%%%%%%
\section{GPU Computing}
Graphical Processing Units (GPUs) can be found on virtually every computer, and offer the possibility of 10x to 50x increase in speed for a large class of algorithms, namely those amenable to SIMD (Single Instruction Multiple Data)  parallelization. \textit{GPU computing}, also known as GPGPU, is a field of study that develops the use of GPUs for general purpose computing rather than for graphical applications. Over the past decade, scientists and engineers have increasingly been taking advantage of GPU computing to enhance the performance of their applications.

GPU computing is often considered to be a heterogeneous system, a mixture of GPUs and CPUs used together to process one or more tasks. Typically, sequential code is implemented on the CPUs, and the computationally-intensive and parallelizable code is relegated to the GPUs.

Because flocking is at its core a massively parallel concept (each member of a flock is treated equally through the algorithm), GPUs seem like a natural fit to improve algorithm performance, opening the door to flocks with tens or hundreds of thousands of boids. Examples of super-flocks would include armies of ants and the migration of butterflies. 

% TODO: need references
%\mmy{NEED REFERENCES: armies of ants, and butterflies}

% GPU device
\subsection{Programmable GPU}
A GPU device is a computing engine\cite{GPUbook} that focuses on performing a massive number of floating-point calculations in parallel. GPUs have a vector architecture that has significant advantages when executing identical operations on arrays whose data is stored sequentially in memory. They have many Arithmetic Logic Units with small control units and a small amount of cache memory. 
To the contrary CPUs, have a large logical control units and a large cache. 

% GPU programming languages
\subsection{GPU Programming Languages}
Since the GPUs have a different architecture than the CPUs, they need to be programmed in a special manner. Although there have been many languages developed for the GPUs over the past decade, the two most prevalent are CUDA and OpenCL. CUDA is the NVIDIA  general purpose application programming interface (API)\cite{CUDAandOpenCL}, while OpenCL was standardized by the Khronos Group\cite{opencl}. OpenCL, which is an extension to the C-language,  enables different computing devices to execute its code, and exchange data; this is not 
possible with CUDA.
 
% Flocking in GPU
\subsection{Flocking in GPU}\label{flockingGPU}
Flocking algorithms can take advantage of GPU-based implementations. Erra et al.\cite{BehaveRT} developed a GPU-based library for autonomous character called BehaveRT.

Other applications of flocking have been also implemented on the GPU. Crowd simulations of more than one million boids have been performed at interactive rates of 25 fps\cite{supermassiveCrowd}. Weiss developed a ClusterFlockGPU algorithm in CUDA; it was possible to run large groups of up to 5,000 instances at 12 fps\cite{SI_GPU}. Clustering of documents is another application of flocking that has also been implemented on the GPU, and achieved a 5x speedup relative to the CPU for 3,000 documents\cite{document2}. 

%%%%%%%%%%%%%
% Educational Games
%%%%%%%%%%%%%
\section{Educational Games}
Games can be classified into multiple categories, such as shooter games, sport games, massively multiplayer games. Any one of these categories can be recast to have an educational component, thus the emergence of educational games. The increase in the number of hours spent in  gameplay may address the poor performance of students in mathematics and the sciences in American schools, when playing educational games. 

 Creating compelling educational games is no easy task. If it is too easy, the students do not learn, if it is too difficult, the students are frustrated and if it is too obvious that the game is meant to teach, the student might not be interested in playing. Ultimately, the game must combine challenge, fun, and provide a learning curve that is adapted to the player.

Thus, we feel that it is important to incorporate general tools such as flocking into public domain game engines to provide game writers increased flexibility to develop software that help students to learn within an environment they are comfortable with. While some games already contain flocking and particle modules, they are most often running on the CPU, so real-time interactivity is not possible, except for small flock sizes. 

% Flocking games
\subsection{Flocking Games}
Steven Woodcock published a code in the book Game Programming Gems in which he considered flocking as a simple technique to simulate group behavior\cite{gems1}. In this code, he implemented four rules: separation, alignment, cohesion, and avoidance.  Flocking is a stateless nature, the state of each boid does not depend on its state at the previous step, thus conserving on memory. Flocking can be used in games to enhance the \textit{illusion of reality} for the user.  In 2001 Woodcock describes an application of flocking inspired by predators and prey\cite{gems2}. 
He also implemented obstacle avoidance.

