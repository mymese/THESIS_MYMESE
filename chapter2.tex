\chapter{RELATED WORK}

In this chapter, we will discuss the different work that has been done in the area of flocking. Results from others researchers are also going to be discussed. 

The chapter goes as follows, first we are going to talk about Swarm Intelligence this is a general area that study swarm behaviors between entities. Then, some results about basic flocking are going to be mentioned follow by a discussion on Flocking and GPU computing. The final section talks about Flocking and Games.

\section{Swarm Intelligence}
Swarms are seen in nature mostly as large groups of small insects \cite{?}. Each entity of a swarm performs a simple role that evolves into a complex behavior as a whole. The emerge of this complex behavior goes beyond swarm since it can also be seen in social animals like birds and fishes. 

% take a picture of a flock to add it here 

Each entity of a swarm will follow a set of simple rules which depends from the entities of the swarm. Every time the rules are changed the environment conditions are evaluated. Since this process is sometimes very simple researches started trying to model it. Therefore, a new discipline in the area of artificial intelligence started to growth.

In 1989 Gerardo Beni and Jing Wang introduced the term Swarm Intelligence (SI) to the Artificial Intelligence (AI) community. SI was born in order to understand the biological insights about the ability of social life insects to solve their everyday problems \cite{BioPrinciplesSI}. As many areas of science, researchers give different definitions to it. XXXX and XXX defined SI as an AI discipline that studies the self-organized behavior of multi-agent systems \cite{?}. 

Lets explain what is self-organized behavior and multi-agent systems are and were are they used for. Self-organized behavior

\section{Basic Flocking}

\subsection{Early Work}
Flocking has been study since XXX. It became very popular after Craig Reynolds published his flocking model called \textit{Boids}\cite{craig1}. He also introduced the name Boid and he defined as \textit{simulated bird-like, "bird-old" objects generically as "boids" even when they represent other sorts of creatures such as schooling fish}. 

In his 1987 paper Reynolds described a model in which he define three rules

\subsection{Current Work}

\section{Flocking with GPU Computing}


\section{Flocking and Games}


