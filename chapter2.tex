\chapter{RELATED WORK}

%%%%%%%%%%%%%
% Introduction
%%%%%%%%%%%%%
\section{Introduction}
Flocking is a nature inspired behavior that can be seen in different social animals. It is most seen in birds, i.e. flock of birds. Craig Reynolds the pioneer of flocking defined as \textit{the result of the interaction between the behaviors of individual birds}\cite{craig1}.Research work related to flocking started to been developed by mid 1980s. 

This chapter we are going to discuss some the related work that have been done in flocking, more general in Swarm Intelligence. The chapter goes as follows, first we are going to talk about Swarm Intelligence this is the discipline that study all swarm behaviors including flocking. Then, some results about basic flocking are going to be mentioned. This is going to include a summary of the early work done by Reynolds and the current work done by other researchers. This is going to be follow by a discussion on Flocking and GPU computing, and the final section describes the relation between Flocking and Games.

%%%%%%%%%%%%%
% Computational Intelligence
%%%%%%%%%%%%%
\section{Computational Intelligence}

% Swarm intelligence
\subsection{Swarm Intelligence} 
Swarms are seen in nature mostly as large groups of small insects \cite{?}. Each entity of a swarm performs a simple role that evolves into a complex behavior as a whole. The emerge of this complex behavior goes beyond swarm since it can also be seen in social animals like birds and fishes. 

% take a picture of a flock to add it here 

Each entity of a swarm will follow a set of simple rules which depends from the entities of the swarm. Every time the rules are changed the environment conditions are evaluated. Since this process is sometimes very simple researches started trying to model it. Therefore, a new discipline in the area of artificial intelligence started to growth.

In 1989 Gerardo Beni and Jing Wang introduced the term Swarm Intelligence (SI) to the Artificial Intelligence (AI) community. SI was born in order to understand the biological insights about the ability of social life insects to solve their everyday problems \cite{BioPrinciplesSI}. As many areas of science, researchers give different definitions to it. XXXX and XXX defined SI as an AI discipline that studies the self-organized behavior of multi-agent systems \cite{?}. 

Lets explain what is self-organized behavior and multi-agent systems are and were are they used for. Self-organized behavior.

% mention ACO and PSO

% Evolutionary computation
\subsection{Evolutionary Computation}

% mention GA


%%%%%%%%%%%%%
% Flocking
%%%%%%%%%%%%%
\section{Flocking}
Flocking became very popular in the late 1980s. Inspired by the behavior of birds, Craig Reynolds developed a behavioral model that simulates self-organizing boids.

% Early work by Craig Reynolds
\subsection{Early Work by Craig Reynolds}
As mentioned before flocking became very popular after Craig Reynolds published his flocking model called \textit{Boids}\cite{craig1}. Reynolds introduced the name Boid as \textit{simulated bird-like, \textbf{bird-old} objects generically as \textbf{boids} even when they represent other sorts of creatures such as schooling fish}. Reynolds model was inspired in the flocking behavior observed on birds. His contributions on this paper made more researchers became interested on studying this naturally-inspired behavior.

Before presenting the model he described the bird mechanisms and the aspects of the physics of the flight. The behaviors of the boids are represented with rules and the internal state of each of them is stored in a data structure. The geometric flight of each boid is the motion along the path it is traveling. Geometric flight relates translation, pitch and yaw. 

Natural flocks flight have balance between  a desire to stay closer to each other and a desire to avoid collisions with their neighbors. This balance lead to the definition of the Boid model. The first rule defined by Reynolds is \textit{collision avoidance} which means that each boid is going to avoid collisions with their nearby flockmates; the second rule is \textit{velocity matching}, this rule attempt to match the velocity of the boids with the nearby flockmates; the last rule is \textit{flock centering} which the boids try to stay closer to their flockmates.

Collision avoidance is based on the relative positions and velocity matching is based only on the velocities, therefore they are complementary. Flock centering make each boid to steer towards the center of the flock. Each boid stores its internal state and evaluate each of the rules individually based on their flockmates. 

In its model the neighborhood is defined as a spherical zone around the boid's local origin. For each run Reynolds initialized the positions, velocities, and various parameters with random values. 

When avoiding collisions with obstacles, the boids reacts depending on the force field that is surrounding it. The boids consider objects that are only in front of it. The naive implementation of the Boids model developed by Reynolds in 1987 was O(N�2). He also did a parallel implementation which was O(N) with respect to the population.

This paper was successful in the developing of an algorithm that simulates independent boids that try to avoid collisions with themselves and with obstacles objects in the environment, and also to stick together.

% Current work
\subsection{Current Work}

%%%%%%%%%%%%%
% GPU Computing
%%%%%%%%%%%%%
\section{GPU Computing}

% GPU device
\subsection{GPU Device}

% GPU programming languages
\subsection{GPU Programming Languages}

% Flocking in GPU
\subsection{Flocking in GPU}

Swarm intelligence algorithms can take the advantage of GPU-based implementations. Weiss developed a ClusterFlockGPU algorithm \cite{SI_GPU}. This algorithm is a swarm intelligence data mining algorithm for partitional cluster analysis based on flocking behaviors. ClusterFlockGPU was implemented in CUDA. 

%%%%%%%%%%%%%
% Educational Games
%%%%%%%%%%%%%
\section{Educational Games}

% Flocking games
\subsection{Flocking Games}



