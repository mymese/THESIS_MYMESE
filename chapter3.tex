\chapter{FLOCKING ALGORITHM}\label{chap3}

% introduction
%\section{Introduction}
The self-organization seen in social animals like birds and fishes is called \textit{flocking}. Flocking behavior is simulated by following a set of simple rules. These rules were introduced by the pioneer of flocking simulations Craig Reynolds\cite{craig1}. The three main rules or main steering behaviors are: \textit{separation}, \textit{alignment}, and \textit{cohesion}. This chapter discusses the different behaviors implemented in our code, which include the three main steering behaviors and three additional behaviors.

% 3 steering behaviors
\section{Three Main Steering Behaviors}
Boid evolution is subject to several behavior rules, each with a corresponding
velocity vector. We discuss some of these rules in this section. 

%. Each rule is associated with a velocity vector that directs the movement of the boid. The velocities combine linearly to 
%velocity vector is scaled by a weight factor and summed to 
%produce a single velocity applied to the boid. 
%Each behavior is evaluated at each time step for each boid.

%More generally, for an arbitrary number of rules  $rule_r$, and scalar weights $w_{rule_r}$, the boid moves in the direction of the velocity
%\begin{equation}
%vel^{k+1} = vel^{k} + \sum_{rule_r} w_{rule_r} {rule_r}
%\end{equation}
	
%The steering behaviors are velocity vectors, they have direction and magnitude. 

% separation
\subsection{Separation}\label{separationsection}
\textit{Separation} can be described as the steering velocity necessary to keep each boid at some specified distance away from its neighbors. Reynolds called this rule \textit{collision avoidance} to indicate that boids try to avoid colliding with nearby flockmates. Later, the name of the rule was changed to \textit{separation}. This steering behavior prevents excessive crowding of boids. \textit{Separation} is a repulsive velocity. Figure~\ref{separationPDF} shows that only flockmates within a certain radius $R_{sep}$ are considered when evaluating this behavior. The red vector is the resulting steering velocity.

% separation figure
\begin{figure}[htbp]
\vspace{16pt}
\begin{center}
\includegraphics[scale=0.85]{figures/separation.pdf}
\caption{Separation steering velocity: the outer circle is the neighborhood of boid $i$, the inner circle is the minimum distance neighborhood, \textit{separation} steers boid $i$ away/near the neighboring boids until a minimum distance between them is met.}
\label{separationPDF}
\end{center}
\end{figure}

The mathematical expression that was implemented is showed in equation~\ref{separationEquation}.

% separation equation
\begin{equation}
\label{separationEquation}
Separation =\frac{1}{M} \sum_{n=1}^{M} \frac{p_i - p_n}{d(p_i,p_n)},
\end{equation}
where $M$ is the number of boids within the minimum distance from boid $i$, $p_i$ is the position of boid $i$, and $d(p_i,p_n)$ is the distance between boid $i$ and neighbor $n$.

%Only neighbors that are within the minimum separation distance are considered. The difference between the positions of boid $i$ and boid $n$ is divided by the distance between boid $i$ and boid $n$ are summed. Then, the sum is divided by the number of boids that were within the minimum distance. The resulting vector corresponds to the \textit{separation} steer of boid $i$ with respect to their nearest flockmates. 

The separation velocity of boid $i$ is a linear combination of unit vectors pointing toward nearest neighbors. Thus each neighboring boid exerts equal influence on boid $i$, with only direction changing. 

% alignment
\subsection{Alignment}
\textit{Alignment} is the steering behavior wherein each boid tries to align its velocity with that its local neighbors. Reynolds initially called this behavior \textit{velocity matching}; later the name was changed to \textit{alignment}. This behavior also prevents crowding. According to this rule, each boid \textit{aligns} its velocity with the average velocities of their flockmates. Figure~\ref{alignmentPDF} shows the corrective steering velocity vector in red.

% alignment figure
\begin{figure}[htbp]
\vspace{16pt}
\begin{center}
\includegraphics[scale=0.85]{figures/alignment.pdf}
\caption{Alignment steering velocity: the circle \textit{R} is the neighborhood of boid $i$, the green vector acting on boid $i$ is its desired velocity, \textit{alignment} matches the heading and speed of boid $i$ with the velocity of its neighbors (taken as an average).}
\label{alignmentPDF}
\end{center}
\end{figure}

\textit{Alignment} is also calculated using the local neighborhood of each boid. This steering behavior was computed using equation~\ref{alignmentEquation}.

% alignment equation
\begin{equation}
\label{alignmentEquation}
Alignment = \left[  \frac{1}{N} \sum_{n=1}^{N} v_n \right ] - v_i
\end{equation}
where $N$ is the number of boids within the search radius $R$ centered at boid $i$ ($R>R_{sep}$). $v_n$ is the velocity of neighbor $n$. The \textit{alignment} corrective velocity is vector in the direction of the desired velocity, which is also the center of mass with respect to the velocities.

% cohesion
\subsection{Cohesion}
\textit{Cohesion} is responsible for steering boids towards the center of the flock. When Reynolds introduced his \textit{Boids} model, he called this behavior \textit{flock centering}, defined as \textit{the attempt to stay close to nearby objects}. The name of the rule changed to \textit{cohesion}, wherein each boid is attracted to the center of the flock. In our case each boid is attracted to the center of mass of its local neighborhood. Figure~\ref{cohesionPDF} shows the \textit{cohesion} steering velocity in red.

% cohesion figure
\begin{figure}[htbp]
\vspace{16pt}
\begin{center}
\includegraphics[scale=0.85]{figures/cohesion.pdf}
\caption{Cohesion steering velocity: the circle \textit{R} is the neighborhood of boid $i$, \textit{cohesion} steers boid $i$ towards the center of its local neighborhood.}
\label{cohesionPDF}
\end{center}
\end{figure}

The formula used for \textit{cohesion} is very similar to the formula used for \textit{alignment}, the only difference is that \textit{cohesion} uses the positions of the boids while \textit{alignment} uses the boid velocities (Equation~\ref{cohesionEquation}),

% cohesion equation
\begin{equation}
\label{cohesionEquation}
Cohesion = \left[  \frac{1}{N} \sum_{n=1}^{N} p_n \right ] - p_i, 
\end{equation}
where $N$ is the number of boids in the local neighborhood of boid $i$ and $p_n$ is the position of the neighbor $n$. The direction of movement for boid $i$ is towards the center of mass of the boids in its local neighborhood. 

% other behaviors
\section{Other Steering Behaviors}\label{otherbehaviors}
Besides the three main steering behaviors just discussed, many additional behaviors have been considered. In this section, we discuss three additional behaviors that we have implemented: \textit{goal}, \textit{avoid}, and \textit{follow the leader}.

% goal
\subsection{Goal}
\textit{Goal} is the steering behavior that attracts boids to a specific location in global space. This location is static. Figure~\ref{goalPDF} shows the action taken when approaching the target. The red vector shows the direction taken by the boid towards the \textit{goal}. The green vector is the desired velocity, while the dotted line is the actual path taken by the boid.

% goal figure
\begin{figure}[htbp]
\vspace{16pt}
\begin{center}
\includegraphics[scale=0.95]{figures/goal.pdf}
\caption{Goal behavior: boid $i$ approaches the target when subjected to the \textit{goal} behavior.}
\label{goalPDF}
\end{center}
\end{figure}

The \textit{goal} behavior adjusts the boid velocity such that it is aligned towards the target. \textit{How it is computed?} Since the target is static, we compute the desired velocity which is the unit vector (from the boid to the target) and set the magnitude of the desired velocity to the maximum speed allowed to approach the target as rapidly as possible. 
%The steering vector is the vector that points from the boid to the desired velocity. 
More succinctly, 

% goal equation
\begin{equation}
\label{goalEquation}
Goal = \left[\frac{(p_t - p_i)}{d(p_t,p_i)} * max_{speed} \right] - v_i
\end{equation}
where $p_i$ and $v_i$ are the position and velocity of boid $i$, respectively, and $p_t$ is the position of the target. Thus the boid approaches the goal without necessarily reaching it, somewhat akin to a moth buzzing around a light bulb.

% avoid
\subsection{Avoidance}
\textit{Avoidance} or \textit{avoid} refers to a boid steering away from a static target in the global space. The approach is the opposite of the \textit{goal} steering. The desired velocity of \textit{avoid} points in the opposite direction of \textit{goal}'s desired velocity. Figure~\ref{avoidPDF} depicts the steering vector in red and the desired velocity in green. As a formula, 
%Equation~\ref{avoidEquation} shows the mathematical formula for 
\textit{avoidance} become

% avoid equation
\begin{equation}
\label{avoidEquation}
Avoidance = -\left[\frac{(p_t - p_i)}{d(p_t,p_i)} * max_{speed} \right] - v_i
\end{equation}

The negative sign indicates that it points in a direction opposite to the velocity in \textit{goal}. \\ \\ 

% flee figure
\begin{figure}[htbp]
\vspace{16pt}
\begin{center}
\includegraphics[scale=0.95]{figures/avoidance.pdf}
\caption{Avoidance behavior: boid $i$ moves away from the target when subject to the \textit{avoid} behavior.}
\label{avoidPDF}
\end{center}
\end{figure}

% leader following
\subsection{Following the Leader}
A single boid (called \textit{leader}) is in charge of the flock. The remaining boids are the \textit{followers}. This behavior is just one of the many complex behaviors that a flock can exhibit as a whole. Followers come in slightly behind the leader, viewed as a target. At the same time, the leader targets a specific location in the global space. 

% leader following figure
\begin{figure}[htbp]
\vspace{16pt}
\begin{center}
\includegraphics[scale=0.75]{figures/leaderFollowing.pdf}
\caption{Leader Following behavior: the followers follow the leader by approaching a point slightly behind the leader. If a follower finds itself in a rectangular area in front of the leader it will move out of the leader's way.}
\label{leaderPDF}
\end{center}
\end{figure}

Generally, the followers tend to stay close to the leader while staying out of its path, as seen in Figure~\ref{leaderPDF}. If a follower is within the rectangular region ahead of the leader it distances itself by clearing a path for the leader to move freely without worry of collision. The leader's movement is  based on Equation~\ref{goalEquation} where $p_t$ is the leader's target position within the global space. The followers that are outside the rectangular region will approach a point slightly behind the leader, as expressed by Equation~\ref{goalEquation}.

% combining behaviors
\section{Combining the Steering Behaviors}
Each of the steering behaviors discussed above returns a velocity vector. These velocities form a linear combination to produce a final boid velocity. Each velocity contribution is controlled by a scalar weight, specified by the user. Equation~\ref{combine} shows the updated velocity for each boid $i$ at time step $k+1$.

% combine equation
\begin{equation}
\label{combine}
vel_i^{k+1} = vel_i^{k} + \sum_{rule_r} w_{rule_r} {rule_r} 
\end{equation}

%\mmy{I do not see avoidance and the other behaviors you implemented}
%The \textit{new} velocity is just the velocity of Equation~\ref{combine} plus the current velocity. 
A first order Euler integration method was used  to calculate the \textit{new} position of each boid: 

% integrate equation
\begin{align}
\label{integrate}
p_i^{k+1} = p_i^k + dt~ v_i
\end{align}
where $p_i^{k+1}$ is the position of boid $i$ at step $k+1$, $v_i$ is the velocity of the boid, and $dt$ is the integration time step. 

The various velocities are recomputed and the boid positions are updated at every time step. 

% algorithms
\section{Algorithms}
In this section, we present three algorithms. The first algorithm computes the different steering rules. The second algorithm combines the rules and updates the boid positions, and the third algorithm updates the FLOCK boids system at each frame of the simulation. 

The following algorithm presents the GPU implementation of each of the steering rules. To save time, one only computes the rules with non-zero scalar weights. 

% flocking algorithm
\begin{algorithm}
\caption{Flocking algorithm to follow Separation, Alignment, Cohesion, Goal, and Avoid steering behaviors.}
\label{flockingAlgorithm}
\begin{algorithmic}
\FOR {each neighbor $j$ of boid $i$}
	\IF{d($pos_i$, $pos_j$) $<=$ searching radius}
	\STATE flockmates++
		\IF{$w_{sep} > 0$}
			\IF{d($pos_i$, $pos_j$) $<=$ minimum distance}
				\STATE nearestFlockmates++
				\STATE s = $pos_i$ - $pos_j$ 
				\STATE s /= d($pos_i$, $pos_j$) 
				\STATE separation += s
			\ENDIF
		\ENDIF
		\IF{$w_{align} > 0$}
			\STATE alignment += $vel_j$
		\ENDIF
		\IF{$w_{coh} > 0$}
			\STATE cohesion += $pos_j$
		\ENDIF
	\ENDIF
\ENDFOR
\IF{$w_{sep} > 0$ \&\& nearestFlockmates $> 0$}
	\STATE separation /= nearestFlockmates
\ENDIF
\IF{$w_{align} > 0$ \&\& flockmates $> 0$}
	\STATE alignment /=  flockmates
	\STATE alignment -= $vel_i$
\ENDIF
\IF{$w_{coh} > 0$ \&\& flockmates $> 0$}
	\STATE cohesion /=  flockmates
	\STATE cohesion -= $pos_i$
\ENDIF
\IF{$w_{goal} > 0$}
	\STATE $pos_t$ = target
	\STATE desiredVel = (normalize($pos_t - pos_i$) * $max_{speed}$) / d($pos_t, pos_i$) 
	\STATE goal = desiredVel - $vel_i$
\ENDIF
\IF{$w_{avoid} > 0$}
	\STATE $pos_t$ = target
	\STATE desiredVel = -(normalize($pos_t - pos_i$) * $max_{speed}$) / d($pos_t, pos_i$) 
	\STATE avoid = desiredVel - $vel_i$
\ENDIF

\end{algorithmic}
\end{algorithm}

In the Algorithm~\ref{flockingAlgorithm}, each rule is evaluated sequentially for each boid. The double loop is first over all boids and then over all the flockmates of those boids. First we determine (and count) the flockmates in the neighborhood covered by the searching radius, which are input into rules \textit{cohesion} and \textit{alignment}. A similar approach was taken for the CPU implementation of Algorithm~\ref{flockingAlgorithm}. 

% combine and integrate
\vspace{32pt}
\begin{algorithm}
\caption{Combine, integrate and check the boundaries.}
\label{combineAlgorithm}
\begin{algorithmic}
\STATE $vel_i^{k+1} = vel_i^{k} + \sum_{rule_r} w_{rule_r} {rule_r} $
\IF{length($vel_i^{k+1}$) $>$ $max_{speed}$}
	\STATE $vel_i^{k+1}$ = normalize($vel_i^{k+1}$) * $max_{speed}$
\ENDIF  
\STATE (Optional) Add an imposed velocity field to the current velocity.
\STATE $pos_i^{k+1}$ = $pos_i^{k}$ + $dt$ * $vel_i^{k+1}$
\STATE Apply periodic boundary conditions to the \textit{new} position.
\end{algorithmic}
\end{algorithm} 
\vspace{32pt}

Algorithm~\ref{combineAlgorithm} presents the steps necessary to update the boid position at each iteration. For each $rule_r$, the respective weighted velocity is computed. Adding the weighted velocities of each rule to the current velocity produces the final velocity vector of boid $i$ at time step $k+1$. First order Euler integration is applied to update the boid positions. Since flocking is an overall behavior we are not very concerned with spatial accuracy. A first order algorithm is sufficient. To maintain the boid's density in our system and avoid the treatment of boundaries, we applied periodic boundary conditions. 
 
% update
\vspace{32pt}
\begin{algorithm}
\caption{Update of each frame of the simulation.}
\label{updateAlgorithm}
\begin{algorithmic}
\STATE Create a FLOCK boids system.
\FOR {each frame}
\STATE Do the nearest neighbor search.
\STATE Compute the rules.
\STATE Integrate over time.
\STATE Render the new position.
\ENDFOR
\end{algorithmic}
\end{algorithm}
\vspace{32pt}

Algorithm~\ref{updateAlgorithm} is a simplified version of the \texttt{updateGPU()} function (see Section~\ref{flocksection}). First a FLOCK boids system is be created, then the boids enter into a continuous loop, only broken once the simulation ends. 
