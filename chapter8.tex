\chapter{FUTURE WORK}\label{chap8}

This thesis presents the work done over a period of eight months and there is still much that one can envision. This chapter is  divided into two sections: the RTPS framework and the Blender RTPS modifier.

\section{RTPS framework}

% more randomize structures to initialize the positions
% realistic field of view when searching for neighbors
The RTPS framework is at the heart of the project and includes the implementation of boid behavior. Since boids behavior derives from nature, they should be initialized with a certain degree of randomness, both in position, velocity, and geometry. In addition, real boids have a limited field of view, which is not implemented in this thesis. Rather, it is assumed that the boid's view is unrestricted. 
%The current implementation for neighbor search does a $360^{\circ}$ field of view search. 
A more realistic approach would limit the field of vision to a cone.  

% leader following - leader behavior
% goal - static and dynamics
% avoid- static and dynamics
% obstacle/collision avoidance
% prey and predator
% fight
Flocking behavior is much more complex than presented in this thesis. There are rules that are also very important that should be combined with the three basic steering behaviors. These additional rules are essential for natural-looking flocks. For example some rules that may be added are: 

\begin{enumerate}
\item{\textit{leader following}: we already coded the follower behavior in this rule; the behavior of the leader  is still under development;}
\item{\textit{goal}: this rule was explained in Section~\ref{otherbehaviors}; a more complex implementation would include the attraction to static and dynamic objects;}
\item{\textit{avoid}: analog to goal, this rule would repeal static and dynamic objects;} 
\item{\textit{obstacle/collision avoidance}: which leads to boids that avoid obstacles that might lie in the path of the boids;}
\item{\textit{fight}: also may be known as predator/prey behavior, which implies boids than can  attack or retreat depending on circumstances.}
\end{enumerate}

These are only some of the rules that we consider to be essential in a boids system in order to get a simulation with a natural appearance.

% terrain, fly and swim behaviors - add gravity and other parameters
Flocks and Boids became more attractive to research scientists after Reynolds publication in 1987\cite{craig1}, in which he presented his Boids model to simulate flocks. These terms, flock and boids, can be extended to more general behavior and more general entities. Thus, we would also like to develop more complex crowding and terrain (or underwater) flocking behavior\cite{supermassiveCrowd}. 
%or  along with swimming or underwater entity behavior. 
%Flocking algorithms have been used previously to simulate terrain and underwater behaviors\cite{supermassiveCrowd}. 

% optimal selection of the parameters using PSO or GA
% evaluation of the rules
Flocking simulations are controlled by several parameters that depend on the physical application. A question often posed concerns the choice of correct parameter values. 
%often arises is \textit{which are the correct parameters?}. 
There exists a series of algorithms that optimize the process of finding the optimal set of parameters of a given simulation. These algorithms include Particle Swarm Optimization, inspired by flocking, and Genetic Algorithms. They are based on Swarm Intelligence and in Evolutionary Algorithms, respectively, and they have been previously applied to flocking\cite{opt_params_GA}.

% remove boids from the flock
Cohesion is the rule responsible for maintaining the entities of the flock closer together. However, on occasion the entities do not follow the rules and stray away from the flock. Alternatively, boids can die. It would therefore be interesting to include the capability of deleting boids from our flock. 

% add more than one flock
% interaction between SPH and FLOCK particles
Flocks may be heterogeneous, meaning that not all entities are from the same type  (e.g., several schools of fish of different types). Having the option to create multiple flocks and make the heterogenous boids interact with one another is another improvement we envision. This idea also could be integrated  with the idea of supporting hybrid systems as a single collection of particles, i.e., interactions between FLOCK boids and SPH particles.

% special rendering methods for boids
The final improvement to consider within the RTPS framework is the development of special rendering methods for boids. 

\section{Blender RTPS modifier}
% meet almost all the specifications already available at the Boids system
Our aim in this thesis was to enhance Blender capabilities by adding particle systems to its Game Engine. This was done by developing a custom modifier to the RTPS framework created in the context of fluid simulation. Since the current implemented flocking behavior in the RTPS library only includes six steering behaviors, the Blender Boids system currently retains more flexibility to create animations with boids, but only outside the Game Engine.  Although our ambition was to duplicate most of the options already available in the boid's simulator, but inside the Game Engine and on the GPU, this was only partially achieved. All that is missing is a few rules, which should be easy to handle. The most difficult problem might be the implementation of efficient  boundary avoidance. 
%in an efficient manner. 

% support other shapes besides box, not depend on the bounding box - emitter object
Another improvement that would improve our new boid simulator is the capability to use different objects as emitters. Currently, the particles are emitted using the bounding box of the emitter object. However, it might be more useful to support not only all the primitive objects of Blender, but also objects or characters custom made by the designer. Along with that we would like to support those custom objects also for rendering, and make all the connections to only not support materials but also to support textures.

There are many more ideas of what can be done to improve or extend what the game designer might desire. We just only mentioned a few.

