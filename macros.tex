\usepackage[usenames,dvipsnames]{color}
%\usepackage[english]{babel}
\usepackage{tabularx}
\usepackage{soul}
\usepackage{xparse}
\usepackage{listings}


%%%%%%%%%%%%%%%
% Show a list of items "todo" or "done" 
% USAGE: 
% \begin{todolist} 
% 	\todo Something not finished
% 	\done Something finished
% \end{todolist} 
\newenvironment{todolist}{%
  \begin{list}{}{}% whatever you want the list to be
  \let\olditem\item
  \renewcommand\item{\olditem \textcolor{red}{(TODO)}: }
  \newcommand\todo{\olditem \textcolor{red}{(TODO)}: }
   \newcommand\done{\olditem \textcolor{ForestGreen}{(DONE)}: }
}{%
  \end{list}
} 
%%%%%%%%%%%%%%%



%%%%%%%%%%%%%%%
% Show a Author's Note
% USAGE: 
% \authnote[Optional footnote message to further clarify note]{The note to your readers}
\DeclareDocumentCommand \authnote { o m }
{%
\IfNoValueTF {#1}
{\textcolor{blue}{Author's Note: \uline{#2}}} 
{\textcolor{blue}{Author's Note: \uline{#2}}\footnote{Comment: #1}}%
{\textcolor{red}{Gordon's Note: \uline{#2}}\footnote{Comment: #1}}%
}
%%%%%%%%%%%%%%%



%%%%%%%%%%%%%%%
% Strike out text that doesn't belong in the paper
% USAGE: 
% \strike[Optional footnote to state why it doesnt belong]{Text to strike out}
\DeclareDocumentCommand \strike { o m }
{%
\IfNoValueTF {#1}
{\textcolor{red}{\sout{#2}}} 
{\textcolor{red}{\sout{#2}}\footnote{Comment: #1}}%
}
%%%%%%%%%%%%%%%

\definecolor{light-gray}{gray}{0.95}

\newcommand{\cbox}[3]{
\ \\
\fcolorbox{#1}{#2}{
\parbox{\textwidth}{
#3
}
}
}

% Setup an environment similar to verbatim but which will highlight any bash commands we have
\lstnewenvironment{unixcmds}[0]
{
%\lstset{language=bash,frame=shadowbox,rulesepcolor=\color{blue}}
\lstset{ %
language=sh,		% Language
basicstyle=\ttfamily,
backgroundcolor=\color{light-gray}, 
rulecolor=\color{blue},
%frame=tb, 
columns=fullflexible,
%framexrightmargin=-.2\textwidth,
linewidth=0.8\textwidth,
breaklines=true,
%prebreak=/, 
  prebreak = \raisebox{0ex}[0ex][0ex]{\ensuremath{\hookleftarrow}},
%basicstyle=\footnotesize,       % the size of the fonts that are used for the code
%numbers=left,                   % where to put the line-numbers
%numberstyle=\footnotesize,      % the size of the fonts that are used for the line-numbers
%stepnumber=2,                   % the step between two line-numbers. If it's 1 each line 
                                % will be numbered
%numbersep=5pt,                  % how far the line-numbers are from the code
showspaces=false,               % show spaces adding particular underscores
showstringspaces=false,         % underline spaces within strings
showtabs=false,                 % show tabs within strings adding particular underscores
frame=single,	                % adds a frame around the code
tabsize=2,	                % sets default tabsize to 2 spaces
captionpos=b,                   % sets the caption-position to bottom
breakatwhitespace=false,        % sets if automatic breaks should only happen at whitespace
}
} { }

% Setup an environment similar to verbatim but which will highlight c++ syntax
\lstnewenvironment{cppcode}[1]
{
%\lstset{language=bash,frame=shadowbox,rulesepcolor=\color{blue}}
\lstset{ %
	backgroundcolor=\color{light-gray},
	rulecolor=\color[rgb]{0.133,0.545,0.133},
	tabsize=4,
	language=[GNU]C++,
%	basicstyle=\ttfamily,
        basicstyle=\scriptsize,
        upquote=true,
        aboveskip={1.5\baselineskip},
        %columns=fullflexible,
        %framexrightmargin=-.1\textwidth,
       %framexleftmargin=6mm,
        showstringspaces=false,
        extendedchars=true,
        breaklines=true,
        prebreak = \raisebox{0ex}[0ex][0ex]{\ensuremath{\hookleftarrow}},
        frame=single,
        showtabs=false,
        tabsize=4,
        showspaces=false,
        showstringspaces=false,
        numbers=left,                   % where to put the line-numbers
        numbers=none,                   % where to put the line-numbers
	numberstyle=\footnotesize,      % the size of the fonts that are used for the line-numbers
	stepnumber=4,                   % the step between two line-numbers. If it's 1 each line 
                                % will be numbered
	firstnumber=#1,
         numbersep=5pt,                  % how far the line-numbers are from the code
        identifierstyle=\ttfamily,
        keywordstyle=\color[rgb]{0,0,1},
        commentstyle=\color[rgb]{0.133,0.545,0.133},
        stringstyle=\color[rgb]{0.627,0.126,0.941},
}
} { }

% Setup an environment similar to verbatim but which will highlight any bash commands we have
\lstnewenvironment{mcode}[1]
{
\lstset{ %
	backgroundcolor=\color{light-gray}, 
	rulecolor=\color[rgb]{0.133,0.545,0.133},
	tabsize=4,
	language=Matlab,
%	basicstyle=\ttfamily,
        basicstyle=\scriptsize,
        upquote=true,
        aboveskip={1.5\baselineskip},
        columns=fullflexible,
        %framexrightmargin=-.1\textwidth,
       %framexleftmargin=6mm,
        showstringspaces=false,
        extendedchars=true,
        breaklines=true,
        prebreak = \raisebox{0ex}[0ex][0ex]{\ensuremath{\hookleftarrow}},
        frame=single,
        showtabs=false,
        showspaces=false,
        showstringspaces=false,
        numbers=left,                   % where to put the line-numbers
	numberstyle=\footnotesize,      % the size of the fonts that are used for the line-numbers
	stepnumber=4,                   % the step between two line-numbers. If it's 1 each line 
                                % will be numbered
	firstnumber=#1,
         numbersep=5pt,                  % how far the line-numbers are from the code
        identifierstyle=\ttfamily,
        keywordstyle=\color[rgb]{0,0,1},
        commentstyle=\color[rgb]{0.133,0.545,0.133},
        stringstyle=\color[rgb]{0.627,0.126,0.941},
}
} { }

\newcommand{\inputmcode}[1]{%
\lstset{ %
	backgroundcolor=\color{light-gray},  %
	rulecolor=\color[rgb]{0.133,0.545,0.133}, %
	tabsize=4, %
	language=Matlab, %
%	basicstyle=\ttfamily,
        basicstyle=\scriptsize, %
        %        upquote=true,
        aboveskip={1.5\baselineskip}, %
        columns=fullflexible, %
        %framexrightmargin=-.1\textwidth,
       %framexleftmargin=6mm,
        showstringspaces=false, %
        extendedchars=true, %
        breaklines=true, %
        prebreak = \raisebox{0ex}[0ex][0ex]{\ensuremath{\hookleftarrow}}, %
        frame=single, %
        showtabs=false, %
        showspaces=false, %
        showstringspaces=false,%
        numbers=left,                   % where to put the line-numbers
	numberstyle=\footnotesize,      % the size of the fonts that are used for the line-numbers
	stepnumber=4,                   % the step between two line-numbers. If it's 1 each line 
                                % will be numbered
         numbersep=5pt,                  % how far the line-numbers are from the code
        identifierstyle=\ttfamily, %
        keywordstyle=\color[rgb]{0,0,1}, %
        commentstyle=\color[rgb]{0.133,0.545,0.133}, %
        stringstyle=\color[rgb]{0.627,0.126,0.941} %
}
\lstinputlisting{#1}%
}

%\lstset{ %
%	backgroundcolor=\color{light-gray}, 
%	rulecolor=\color[rgb]{0.133,0.545,0.133},
%	tabsize=4,
%	language=Matlab,
%%	basicstyle=\ttfamily,
%        basicstyle=\scriptsize,
%        upquote=true,
%        aboveskip={1.5\baselineskip},
%        columns=fullflexible,
%        %framexrightmargin=-.1\textwidth,
%       %framexleftmargin=6mm,
%        showstringspaces=false,
%        extendedchars=true,
%        breaklines=true,
%        prebreak = \raisebox{0ex}[0ex][0ex]{\ensuremath{\hookleftarrow}},
%        frame=single,
%        showtabs=false,
%        showspaces=false,
%        showstringspaces=false,
%        numbers=left,                   % where to put the line-numbers
%	numberstyle=\footnotesize,      % the size of the fonts that are used for the line-numbers
%	stepnumber=4,                   % the step between two line-numbers. If it's 1 each line 
%                                % will be numbered
%	firstnumber=#1,
%         numbersep=5pt,                  % how far the line-numbers are from the code
%        identifierstyle=\ttfamily,
%        keywordstyle=\color[rgb]{0,0,1},
%        commentstyle=\color[rgb]{0.133,0.545,0.133},
%        stringstyle=\color[rgb]{0.627,0.126,0.941},
%}


\newcommand{\Laplacian}[1]{\nabla^2 #1}

% set of all nodes received and contained on GPU
\newcommand{\setAllNodes}[0]{\mathcal{G}}
% set of stencil centers on GPU
\newcommand{\setCenters}[0]{\mathcal{Q}}
% set of stencil centers with nodes in \setDepend
\newcommand{\setBoundary}[0]{\mathcal{B}}
% set of nodes received by other GPUs
\newcommand{\setDepend}[0]{\mathcal{R}}
% set of nodes sent to other GPUs
\newcommand{\setProvide}[0]{\mathcal{O}}


\newcommand{\toprule}[0]{\hline}
\newcommand{\midrule}[0]{\hline\hline}
\newcommand{\bottomrule}[0]{\hline}

\newcolumntype{C}{>{\centering\arraybackslash}b{1in}}
\newcolumntype{L}{>{\flushleft\arraybackslash}b{1.5in}}
\newcolumntype{R}{>{\flushright\arraybackslash}b{1.5in}}
\newcolumntype{D}{>{\flushright\arraybackslash}b{2.0in}}
\newcolumntype{E}{>{\flushright\arraybackslash}b{1.0in}}

\DeclareSymbolFont{AMSb}{U}{msb}{m}{n}
\DeclareMathSymbol{\N}{\mathbin}{AMSb}{"4E}
\DeclareMathSymbol{\Z}{\mathbin}{AMSb}{"5A}
\DeclareMathSymbol{\R}{\mathbin}{AMSb}{"52}
\DeclareMathSymbol{\Q}{\mathbin}{AMSb}{"51}
\DeclareMathSymbol{\PP}{\mathbin}{AMSb}{"50}
\DeclareMathSymbol{\I}{\mathbin}{AMSb}{"49}
\DeclareMathSymbol{\C}{\mathbin}{AMSb}{"43}

%%%%%% VECTOR NORM: %%%%%%%
\newcommand{\vectornorm}[1]{\left|\left|#1\right|\right|}
%\renewcommand{\vec}[1]{ \textbf{#1} }
%%%%%%%%%%%%%%%%%%%%%%

%%%%%%% THM, COR, DEF %%%%%%%
%\newtheorem{theorem}{Theorem}[section]
%\newtheorem{lemma}[theorem]{Lemma}
%\newtheorem{proposition}[theorem]{Proposition}
%\newtheorem{corollary}[theorem]{Corollary}
%\newenvironment{proof}[1][Proof]{\begin{trivlist}
%\item[\hskip \labelsep {\bfseries #1}]}{\end{trivlist}}
%\newenvironment{definition}[1][Definition]{\begin{trivlist}
%\item[\hskip \labelsep {\bfseries #1}]}{\end{trivlist}}
%\newenvironment{example}[1][Example]{\begin{trivlist}
%\item[\hskip \labelsep {\bfseries #1}]}{\end{trivlist}}
%\newenvironment{remark}[1][Remark]{\begin{trivlist}
%\item[\hskip \labelsep {\bfseries #1}]}{\end{trivlist}}
%\newcommand{\qed}{\nobreak \ifvmode \relax \else
%      \ifdim\lastskip<1.5em \hskip-\lastskip
%      \hskip1.5em plus0em minus0.5em \fi \nobreak
%      \vrule height0.75em width0.5em depth0.25em\fi}
%%%%%%%%%%%%%%%%%%%%%%

%
%\usepackage[algochapter]{algorithm2e}
%\usepackage[usenames]{color}
% colors to show the corrections
\newcommand{\red}[1]{\textbf{\textcolor{red}{#1}}}
\newcommand{\blue}[1]{\textbf{\textcolor{blue}{#1}}}
\newcommand{\cyan}[1]{\textbf{\textcolor{cyan}{#1}}}
\newcommand{\green}[1]{\textbf{\textcolor{green}{#1}}}
\newcommand{\magenta}[1]{\textbf{\textcolor{magenta}{#1}}}
\newcommand{\orange}[1]{\textbf{\textcolor{orange}{#1}}}
%%%%%%%%%% DK DK
% comments between authors
\newcommand{\toall}[1]{\textbf{\green{@@@ All: #1 @@@}}}
\newcommand{\toevan}[1]{\textbf{\red{*** Evan: #1 ***}}}
\newcommand{\toe}[1]{\textbf{\red{*** Evan: #1 ***}}}

\newcommand{\EDIT}[2]{\cyan{#1}}

\newcommand{\mmy}[1]{\textbf{\blue{*** To Myrna: #1 ***}}}
\newcommand{\toi}[1]{\textbf{\magenta{*** Ian: #1 ***}}}
\newcommand{\togordon}[1]{\textbf{\blue{*** Gordon: #1 ***}}}
\newcommand{\gee}[1]{{\bf{\blue{{\em #1}}}}}
\newcommand{\old}[1]{}
\newcommand{\del}[1]{***#1*** }



% \DeclareMathOperator{\Sample}{Sample}
%\let\vaccent=\v % rename builtin command \v{} to \vaccent{}
%\renewcommand{\vec}[1]{\ensuremath{\mathbf{#1}}} % for vectors
\newcommand{\gv}[1]{\ensuremath{\mbox{\boldmath$ #1 $}}} 
% for vectors of Greek letters
\newcommand{\uv}[1]{\ensuremath{\mathbf{\hat{#1}}}} % for unit vector
\newcommand{\abs}[1]{\left| #1 \right|} % for absolute value
\newcommand{\avg}[1]{\left< #1 \right>} % for average
\let\underdot=\d % rename builtin command \d{} to \underdot{}
\renewcommand{\d}[2]{\frac{d #1}{d #2}} % for derivatives
\newcommand{\dd}[2]{\frac{d^2 #1}{d #2^2}} % for double derivatives
\newcommand{\pd}[2]{\frac{\partial #1}{\partial #2}} 
% for partial derivatives
\newcommand{\pdd}[2]{\frac{\partial^2 #1}{\partial #2^2}} 
\newcommand{\pdda}[3]{\frac{\partial^2 #1}{\partial #2 \partial #3}} 
% for double partial derivatives
\newcommand{\pdc}[3]{\left( \frac{\partial #1}{\partial #2}
 \right)_{#3}} % for thermodynamic partial derivatives
\newcommand{\ket}[1]{\left| #1 \right>} % for Dirac bras
\newcommand{\bra}[1]{\left< #1 \right|} % for Dirac kets
\newcommand{\braket}[2]{\left< #1 \vphantom{#2} \right|
 \left. #2 \vphantom{#1} \right>} % for Dirac brackets
\newcommand{\matrixel}[3]{\left< #1 \vphantom{#2#3} \right|
 #2 \left| #3 \vphantom{#1#2} \right>} % for Dirac matrix elements
\newcommand{\grad}[1]{\gv{\nabla} #1} % for gradient
\let\divsymb=\div % rename builtin command \div to \divsymb
\renewcommand{\div}[1]{\gv{\nabla} \cdot #1} % for divergence
\newcommand{\curl}[1]{\gv{\nabla} \times #1} % for curl
\let\baraccent=\= % rename builtin command \= to \baraccent
\renewcommand{\=}[1]{\stackrel{#1}{=}} % for putting numbers above =
\newcommand{\diffop}[1]{\mathcal{L}#1}
\newcommand{\boundop}[1]{\mathcal{B}#1}
\newcommand{\rvec}[0]{{\bf r}}

\newcommand{\Interior}[0]{\Omega}
\newcommand{\Boundary}[0]{\partial \Omega}

\newcommand{\on}[1]{\hskip1.5em \textrm{ on } #1}

\newcommand{\gemm}{\texttt{GEMM}}
\newcommand{\trmm}{\texttt{TRMM}}
\newcommand{\gesvd}{\texttt{GESVD}}
\newcommand{\geqrf}{\texttt{GEQRF}}


\newcommand{\minitab}[2][l]{\begin{tabular}{#1}#2\end{tabular}}
\newcommand{\comm}[1]{\textcolor{red}{\textit{#1}}}

\newcommand{\nfrac}[2]{
\nicefrac{#1}{#2}
%\frac{#1}{#2}
}
