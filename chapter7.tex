\chapter{CONCLUSION}\label{conclusionChapter}

% Conclusion: what lessons have you learned from the experience? Difficulties, what things should/could be done differently to push the state of the art?

The RTPS library was extended by adding another system, named FLOCK, specialized to our flocking system. The flocking algorithm followed by the FLOCK system was implemented to run on both CPU and GPU. The GPU version was implemented using OpenCL to ensure portability between graphics cards. The FLOCK system computes the three basic steering behaviors at the root of all flocking algorithms: \textit{separation}, \textit{alignment}, and \textit{cohesion}. 

The Blender Game Engine was enhanced by adding a custom modifier called RTPS. This modifier links the RTPS library with the Blender Game Engine. We created the UI of the RTPS modifier for the FLOCK system. The main contribution of this thesis was in extending the SPH particle framework in Blender to include flocking behavior.   

The performance of our code is very important. A comparison between our RTPS GPU implementation and Blender Boids system (which is available only outside the Blender Game Engine) was performed. We found that our library outperforms Blender capabilities at all times by at least an order of magnitude. Of course, rendering speed depended on flock size and on the number of boids inside the searching radius. For example, a flock of more than 130K boids was rendered at a rate of 30 fps.
